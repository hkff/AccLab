\section{Getting started}

\subsection{Install AccLab}
To use the ltl prover, you need to put the following executable files (tspass, fotl-translate) in
\texttt{tools/your\_platform/ (linux/mac/win)}
TSPASS binaries are provided for linux x64 and mac x64 in the folder \texttt{tools/\_platformeName\_/}.
For other platformes you have to compile tspass source code.
The last version of TSPASS can be found in :
{\small{<http://lat.inf.tu-dresden.de/~michel/software/tspass/>}}. The source code for TSPASS version 0.95-0.17
is provided with this tool.
AAL Syntax highlighting modes for emacs, intellij, nano and ace, can be found in \texttt{tools/utils/}.
If you want to run aalc using a symbolic link you need to set the environment variable \texttt{ACCLAB\_PATH} :
\texttt{export ACCLAB\_PATH=<AccLab\_install\_dir>}. You need python3.4.0 or greater.


%%%%%%%%%%
%% aalc
%%%%%%%%%%
\subsection{Using AAL compiler "aalc"}
{\lstset{style=shell}
\begin{lstlisting}[caption={aalc options}]
root@root/:$ python aalc
aalc 
  -h    --help                   display this help and exit
  -i    --input                  the input file
  -i    --output                 the output file
  -c    --compile                compile the file, that can be loaded after using -l
  -m    --monodic                apply monodic check on aal file
  -s    --shell                  run a shell after handling aal program
  -k    --check                  perform a verbose check
  -l    --load                   load a compiled aal file (.aalc) and run a shell
  -t    --fotl                   translate the aal program into FOTL
  -r    --reparse                reparse tspass file
  -r    --recompile              recompile the external files
  -b    --no-colors              disable colors in output
  -x    --compile-stdlib         compile the standard library
  -d    --hotswap                enable hotswaping (for development only)
  -a    --ast                    show ast tree
  -u    --gui [port]             run the gui on the specified port (default 8000)
\end{lstlisting}
}

%%%%%%%%%%%%%%%%%%
%% first aal prog
%%%%%%%%%%%%%%%%%%
\subsection{Writing your first AAL program}
Let consider the following senario, we have three actors :
\begin{itemize}
    \item cloud storage service : let call it \texttt{css} which is a cloud service provider
    \item alice and bob : an end users that uses css service
\end{itemize}

The \texttt{css} offers the following services : read (a user reads some data form css server),
store (a user stores some data into css server), delete (a user deletes some data from css server).
\texttt{css} allows users to read/store/delete only their data on his server, and don't allow them to read other
customers data. \texttt{css} can also read and delete any data from his server.

Alice want to check if \texttt{css} policy respect her privacy. Typically she want to know if she is allowed to
performs some actions and if bob can read here data.


\begin{enumerate}[a.]

    \item \textbf{Declaring actors} : first we need to declare our actors
\begin{lstlisting}
// Agents declaration
AGENT alice
AGENT bob
AGENT css
\end{lstlisting}

    \item \textbf{Declaring services} the services used are :
\begin{lstlisting}
SERVICE read
SERVICE store
SERVICE delete
\end{lstlisting}

    \item \textbf{Linking services and actors}
\begin{lstlisting}
AGENT alice TYPES(Actor) REQUIRED(read store delete) PROVIDED()
AGENT bob   TYPES(Actor) REQUIRED(read store delete) PROVIDED()
AGENT css   TYPES(Actor) REQUIRED() PROVIDED(read store delete)
\end{lstlisting}


    \item \textbf{Defining policies}
\begin{lstlisting}
/*
 * Cloud storage service provider policy
 */
CLAUSE css_policy (
    FORALL d:data FORALL a:Actor

    // Allow users to read their data
    IF (d.subject == a) THEN {
        PERMIT a.read[css](d)
    } AND

    // Deny access to read other
    IF (d.subject != a) THEN {
        DENY a.read[css](d)
    } AND

    // Allow css to read/delete stored data
    PERMIT css.read[css](d) AND
    PERMIT css.delete[css](d)
)

/*
 * Alice's preferences
 */
CLAUSE alice_policy (
    FORALL d:data
    // Alice want to be able to read all her data stored on css
    IF (d.subject == alice) THEN {
        PERMIT alice.read[css](d)
    }
)
\end{lstlisting}

    \item \textbf{Writing checks}
Now we want to check if \texttt{Alice}'s privacy preferences are respected by the \texttt{css} policy.
To do this, we can call the macro \texttt{validate} and passing the the clauses names as arguments.
Important : Note that the order of arguments is important.
\begin{lstlisting}
CALL validate("css_policy" "alice_pref")
\end{lstlisting}

\end{enumerate}


\subsection{Running the program}

\begin{itemize}
	\item Run the AAL program
{\lstset{style=shell}
\begin{lstlisting}
root@root/:$ python aalc -i examples/tuto0.aal

------------------------- Monodic check -------------------------
Monodic check passed !
------------------------- Starting Validity check -------------------------
c1 : css_policy
c2 : alice_pref
----- Checking c1 & c2 consistency :
  -> Satisfiable
----- Checking c1 => c2 :
  -> Satisfiable
----- Checking ~(c1 => c2) :
  -> Unsatisfiable

[VALIDITY] Formula is valid !
------------------------- Validity check End -------------------------

File : examples/tuto0.aal

Execution time : 0.24277639389038086
\end{lstlisting}

Here the result of

	\item Perform an detailed check
\begin{lstlisting}
root@root/:$ python aalc -i examples/tuto0.aal -k

------------------------- Start Checking -------------------------

** DECLARATIONS
[DECLARED AGENTS]   : 3
[DECLARED SERVICES] : 6
[DECLARED DATA]     : 0
[DECLARED TYPES]    : 10

*** Forwards references check
[AGENTS]   : 0
[SERVICES] : 0
[DATA]     : 0
[TYPES]    : 0

** LOADED libraries
[LIBS] : 2

** CLAUSES
[CLAUSES] : 2
Monodic test :
|css_policy | Formula is monodic ! |
|alice_pref | Formula is monodic ! |
-------------------------- Checking End -------------------------
\end{lstlisting}


	\item Perform monodic test on all clauses :
\begin{lstlisting}
root@root/:$ python aalc -i examples/tuto0.aal -m

------------------------- Start Checking -------------------------
|css_policy | Formula is monodic ! |
|alice_pref | Formula is monodic ! |
-------------------------- Checking End -------------------------
\end{lstlisting}


	\item Translate AAL program into FOTL (in tspass syntax):
\begin{lstlisting}
root@root/:$ python aalc -i examples/tuto0.aal -t

------------------------- FOTL Translation start -------------------------

%%%%%%%%% START EVN %%%%%%%%%%%
%%% Types knowledge
data(CTS)  &
Actor(CTS)  &
DataSubject(CTS)  &
DataController(CTS)  &
DataProcessor(CTS)  &
DwDataController(CTS)  &
Auditor(CTS)  &
CloudProvider(CTS)  &
CloudCustomer(CTS)  &
EndUser(CTS)  &


%%% Action authorizations
(![x, y, z] (read(x, y, z) => Pread(x, y, z))) &
(![x, y, z] (store(x, y, z) => Pstore(x, y, z))) &
(![x, y, z] (delete(x, y, z) => Pdelete(x, y, z))) &
(![x, y, z] (read(x, y, z) => Pread(x, y, z))) &
(![x, y, z] (write(x, y, z) => Pwrite(x, y, z))) &
(![x, y, z] (audit(x, y, z) => Paudit(x, y, z))) &


%%% Actors knowledge
Actor(alice)  &
Actor(bob)  &
Actor(css)  &

%%%%%%%%% END EVN %%%%%%%%%%%

%% Clause : css_policy
always(![d] ( data(d) & (![a] ( Actor(a) & (((( (subject(d, a) => Pread(a, css, d)) &
(~subject(d, a) => ~Pread(a, css, d))) & Pread(css, css, d)) & Pdelete(css, css, d)))))) )
%% Clause : alice_pref
always(![d] ( data(d) & ( (subject(d, alice) => Pread(alice, css, d)))) )


-------------------------- FOTL Translation end --------------------------
\end{lstlisting}
}
\end{itemize}


%%%%%%%%%%%%%%%%%%
%% Core libs
%%%%%%%%%%%%%%%%%%
\subsection{Using core libraries}
You can load external AAL files using \texttt{LOAD "aal\_file"}(without the extension) 

\paragraph{core.types} Contains the basic types declarations (Actor, DataSubject, DataController, DataProcessor, ...)
\begin{lstlisting}
LOAD "core.types"
\end{lstlisting}

\paragraph{core.macros} Contains some useful macros.
\begin{lstlisting}
// Loading lib
LOAD "core.macros"
\end{lstlisting}




%%%%%%%%%%%%%%%%%%
%% Advenced checks
%%%%%%%%%%%%%%%%%%
\subsection{Advanced checks}

\begin{lstlisting}
/*
 * Alice's preferences
 */
CLAUSE alice_policy (
    FORALL d:data
    // Alice want to be able to read all her data stored on css
    IF (d.subject == alice) THEN {
        PERMIT alice.read[css](d)

        // Bob cannot read Alice's data
        DENY bob.read[css](d)
    }
)
\end{lstlisting}
The previous call to validate macro will gives : Satisfiable.

Why ? Because the predicate subject is not exclusive :
subject of d can be alice and bob at the same time.

A simple way to fix it is to add the condition that the subject of d is not bob :
\begin{lstlisting}
    IF (d.subject == alice AND d.subject != bob) THEN {
    ....
\end{lstlisting}


Or we can to add the following condition manually to our check :
\begin{lstlisting}
(![f] (subject(f, alice) => ~subject(f, bob))) &
\end{lstlisting}

The construction CHECK allows you to write directly FOTL formula mixed with somme AAL constructions.

\begin{itemize}
    \item @verbose (print the generated formula)
    \item @buildenv (build the environment, which is a set of preconditions generated from the AAL program)
    \item clause(c) : get the fotl translation of clause "c"
    \item clause(c).ue : get the fotl translation of usage part of the clause "c"
    \item clause(c).ae : get the fotl translation of audit part of the clause "c"
    \item clause(c).re : get the fotl translation of rectification part of the clause "c"
    \item APPLY chk() : call the check "chk"
\end{itemize}

\begin{lstlisting}
CHECK c1() {
% Comments in Checks starts with '%'

% Enable verbose mode
@verbose
~(
    % Build the environment
    @buildenv

    % Add extra condition
    (![f] ((subject(f, alice) )=> ~subject(f, bob))) &

    % The check ~ P => U
    (clause(css_policy))
    =>
    (clause(alice_pref))
)
}
APPLY c1()
\end{lstlisting}
The result is Unsatisfiable so the formula is valide.

%%%%%%%%%%%%%%%%%%
%% Using shell
%%%%%%%%%%%%%%%%%%
\subsection{Using the shell}
The shell is a useful tool for developing :

{
\lstset{style=shell}

\begin{itemize}
	\item Run the shell.
\begin{lstlisting}
root@root/:$ python aalc -i examples/tuto0.aal -s
....
shell >
\end{lstlisting}

  \item Type help to show the shell help.
\begin{lstlisting}
shell >help
  Shell Help
 - call(macro, args)    call a macro where /
                         *macro : is the name of the macro
                         *args : a list of string; <<ex :["'args1'", "'args2'", ..."'argsN'"]>>
 - clauses()            show all declared clauses in the loaded aal program
 - macros()             show all declared macros in the loaded aal program
 - load(lib)            load the librarie lib
 - quit / q             exit the shell
 - help / h / man()     show this help
 - self                 the current compiler instance of the loaded aal program
 - aalprog              the current loaded aal program
 - man(arg)             print the help for the given arg
 - hs(module)           hotswaping : reload the module
 - r()                  hot-swaping the shell
\end{lstlisting}

  \item Here an example, we print all clauses in the AAL program.
\begin{lstlisting}
shell> clauses()
css_policy alice_pref

\end{lstlisting}

  \item self variable refers to the compiler instance.
\begin{lstlisting}
shell> self
<AALCompiler.AALCompilerListener object at 0x7f8b00ce8630>
\end{lstlisting}

    \item man can be called on any element, it will show its documentation.
\begin{lstlisting}
shell> man(self)
printing manual for <class 'AALCompiler.AALCompilerListener'>
Manual for aal compiler visitor
 - Attributes
         -  aalprog      Get the AAL program instance
         -  file         The AAL source file
         -  libs         Show the loaded libraries
         -  libsPath     Print the standard lib path
 - Methods
         - load_lib(lib_name)    Load an aal file
         - clause(clauseId)      Lookup for clause cluaseId
         - show_clauses()        Show all clauses (names
         - get_clauses()         Get all clauses (objects)
         - get_macros()          Get all macros (objects)

\end{lstlisting}

    \item AAL program instance
\begin{lstlisting}
shell> man(aalprog)
printing manual for <class 'AALMetaModel.m_aalprog'>

    AAL program class.
    Note that clauses and macros extends a declarable type, but are not in the declarations dict

    Attributes
        - clauses: a list that contains all program clauses
        - declarations: a dictionary that contains lists of typed declarations
        - comments: a list that contains program s comment
        - macros: a list that contains program s macros declarations
        - macroCalls: a list that contains program s comment
\end{lstlisting}

    \item Calling a macro
\begin{lstlisting}
call("validate", ["'css_policy'", "'alice_pref'"])
------------------------- Monodic check -------------------------
Monodic check passed !
------------------------- Starting Validity check -------------------------
c1 : css_policy
c2 : alice_pref
----- Checking c1 & c2 consistency :
  -> Satisfiable
----- Checking c1 => c2 :
  -> Satisfiable
----- Checking ~(c1 => c2) :
  -> Satisfiable
:: Solving trigger

[VALIDITY] Formula is not valid !
------------------------- Validity check End -------------------------
\end{lstlisting}

    \item Defining a new macro
\begin{lstlisting}
shell> self.new_macro("toto", ["p"], """print(p)""")
shell> call("toto", ["'4'"])
4
\end{lstlisting}


  \item hotswaping commands are used for debugging purpose only.
  r() command allows you to reload the shell without exiting it after source code modification.

  \item hs(module) reloading other modules after source code modification without exiting.
  ! IMORTANT : to use hotswaping properly you must enable it explicitly in aalc arguments
  \texttt{-d / --hotswap},

\end{itemize}
}
