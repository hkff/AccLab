\section{Getting started}

\subsection{Install AccLab}
\begin{itemize}
  \item To use the ltl prover, you need to put the following executable files (tspass, fotl-translate) in tools/your\_platform/ (linux/mac/win)
  \item Basic run : python aalc.py (you need python3.4.0 or greater)
  \item Run an AAL file : python aalc.py -i testfile.aal
\end{itemize}


%%%%%%%%%%
%% aalc
%%%%%%%%%%
\subsection{Using AAL compiler "aalc"}
{
\lstset{style=shell}
\begin{lstlisting}[caption={aalc options}]
aalc 
  -h    --help                   display this help and exit
  -i    --input                  the input file
  -i    --output                 the output file
  -c    --compile                compile the file, that can be loaded after using -l
  -m    --monodic                apply monodic check on aal file
  -s    --shell                  run a shell after handling aal program
  -k    --check                  perform a verbose check
  -l    --load                   load a compiled aal file (.aalc) and run a shell
  -t    --ltl                    translate the aal program into FOTL
  -r    --reparse                reparse tspass file
  -r    --recompile              recompile the external files
  -b    --no-colors              disable colors in output
  -x    --compile-stdlib         compile the standard library
  -d    --hotswap                enable hotswaping (for development only)
  -a    --ast                    show ast tree
\end{lstlisting}
}

%%%%%%%%%%%%%%%%%%
%% first aal prog
%%%%%%%%%%%%%%%%%%
\subsection{Writing your first AAL program}
Let consider the following senario, we have three actors :
\begin{itemize}
    \item cloud storage service : let call it \texttt{css} which is a cloud service provider
    \item alice and bob : an end users that uses css service
\end{itemize}

The \texttt{css} offers the following services : read (a user reads some data form css server),
store (a user stores some data into css server), delete (a user deletes some data from css server).
\texttt{css} allows users to read/store/delete only their data on his server, and don't allow them to read other
customers data. \texttt{css} can also read and delete any data from his server.

Alice want to check if \texttt{css} policy respect her privacy. Typically she want to know if she is allowed to
performs some actions and if bob can read here data.


\begin{enumerate}[a.]

    \item \textbf{Declaring services}
The services are the
\begin{lstlisting}
SERVICE read
SERVICE store
SERVICE delete
\end{lstlisting}

    \item \textbf{Declaring actors} : first we need to declare our actors
\begin{lstlisting}
// Agents declaration
AGENT alice
AGENT bob
AGENT css
\end{lstlisting}

    \item \textbf{Linking services and actors}
First we need to declare our actors
\begin{lstlisting}
AGENT alice TYPES() REQUIRED(read store delete) PROVIDED()
AGENT bob   TYPES() REQUIRED(read store delete) PROVIDED()
AGENT css   TYPES() REQUIRED() PROVIDED(read store delete)
\end{lstlisting}


    \item \textbf{Defining policies}
First we need to declare our actors
\begin{lstlisting}
/*
 *
 */
CLAUSE css_policy (

)

/*
 *
 */
CLAUSE alice_policy (

)
\end{lstlisting}

    \item \textbf{Writing checks}
First we need to declare our actors
\begin{lstlisting}
CALL validate("'css_policy'" "'alice_pref'")
\end{lstlisting}
\end{enumerate}


\subsection{Running the program}
{
\begin{itemize}
	\item Run an AAL file (by default, it )
\begin{lstlisting}
aalc -i examples/tuto0.aal

Execution time : 0.26976990699768066
\end{lstlisting}

	\item Perform an detailed check
\begin{lstlisting}
aalc -i tests/tuto1.aal -k

\end{lstlisting}


	\item Perform monodic test on all clauses :
\begin{lstlisting}
aalc -i tests/tuto1.aal -m

------------------------- Checking Monodic test -------------------------
Formula is monodic !
-------------------------- Checking Monodic End -------------------------
\end{lstlisting}


	\item Translate aal program into FOTL (in tspass syntax):
\begin{lstlisting}
aalc -i tests/tuto1.aal -t

------------------------- FOTL Translation start -------------------------
![d] ![a](((((((PERMIT(read, kim, cloudX, d) & PERMIT(read, kim, cloudX, d)) & PERMIT(write, kim, cloudX, d)) & PERMIT(delete, kim, cloudX, d)) & PERMIT(sensors, cloudX, kim, d)) & (~threeYears => Adelete(cloudX, d))) &  ( Aread(a, cloudX, d) => sometime(Anotify(cloudX, kim)))))  ![a] ![d] ![b](((((((((PERMIT(read, a, cloudX, d) & PERMIT (read, a, cloudX, d)) & PERMIT(write, a, cloudX, d)) & PERMIT(delete, a, cloudX, d)) & PERMIT(sensors, cloudX, a, d)) & PERMIT(storage, cloudX, cloudY, d)) & (~twoYears => Adelete(cloudX, d))) &  (Aread(b, cloudX, d) => sometime(Anotify(cloudX, a)))) &  (Astorage(cloudX, cloudY, d) => sometime(Alog(cloudX, cloudX))))) 
-------------------------- FOTL Translation end --------------------------
\end{lstlisting}

\end{itemize}
}




\subsection{Using core libraries}
You can load external AAL files using \texttt{LOAD "aal\_file"}(without the extension) 

\paragraph{core.macros} Contains the basic types declarations (DataSubject, DataController, DataProcessor, ...)
\begin{lstlisting}
// Loading libraries
LOAD "core.types"
\end{lstlisting}



\paragraph{core.macros} Contains some basic macros.
\begin{lstlisting}
// Loading libraries
LOAD "core.macros"

/** ltl check**/
CALL ltl_check()

// Checking validity c1 => c2
CALL validate(c1 c2) (

//
CALL resolve(c1 c2)
'NOT YET IMPLEMENTED'

// Show the loaded libraries used in the current AAL program
CALL show_libs()

// Translate 
CALL ltl(c)

CALL show_clause(c)

CALL to_natural(c)
\end{lstlisting}

\paragraph{core.eu} Contains the basic types
\begin{lstlisting}
// Loading libraries
LOAD "core.eu"

/** Show all obligations **/
CALL obligations() 

/* Result :

Obligations list : (L) Legal / (C) Contractual / (E) Ethical 

 L  Obligation 1-3      : Informing about processing, purposes and recipients. 
 L  Obligation 4        : Informing about rights. 
 L  Obligation 5        : Data collection purposes. 
 L  Obligation 6        : The right to access, correct and delete personal data. 
 L  Obligation 7        : Data storage period. 
 L  Obligation 8,11-12  : Security and privacy measures. 
 L  Obligation 9-10     : Rules for data processing by providers. 
 L  Obligation 13-15    : Consent to processing. 
 L  Obligation 16       : Informing DPAs. 
 C  Obligation 17       : Informing about the use of sub-processors. 
 C  Obligation 18       : Security breach notification. 
 C  Obligation 19-20    : Evidence on data processing and data deletion. 
 C  Obligation 21       : Data location. 
 E  Obligation 22       : Informing about personal data processing. 
 E  Obligation 23       : Personal data minimization. 
 E  Obligation 24       : Privacy-by-default. 
 E  Obligation 25       : Specifying user preferences. 
 E  Obligation 26       : Monitoring of data practices. 
 E  Obligation 27       : Compliance with user preferences. 
 E  Obligation 28       : Compliance with privacy policies. 
 E  Obligation 29-30    : Informing about policy violations and privacy preferences violations. 
 E  Obligation 31       : Remediation in case of incidents. 
*/



/** Checking if the clauses respect **/
CALL obligation18()

/* Obligation 18: Security breach notification. */
  -> No notification in clause kim_policy s rectification at line 15
  -> No notification in clause cloudX_policy s rectification at line 30

\end{lstlisting}




\subsection{Using the shell}
The shell is a useful tool for developing 


{
\lstset{style=shell}

\begin{itemize}
	\item Run the shell.
\begin{lstlisting}
aalc -i tests/tuto2.aal -s

/* Result : 
shell >

*/
\end{lstlisting}

  \item Type help to show the shell help.
\begin{lstlisting}
  Shell Help
 - call(macro, args)   call a macro where /
         *macro : is the name of the macro
         *args : a list of string; << ex : ["'args1'", "'args2'", ..."'argsN'"] >>
 - clauses()           show all declared clauses in the loaded aal program
 - macros()            show all declared macros in the loaded aal program
 - quit / q            exit the shell
 - help / h / man()    show this help
 - self                the current compiler instance of the loaded aal program
 - aalprog             the current loaded aal program 
 - man(arg)            print the help for the given arg
 - hs(module)          hotswaping : reload the module
 - r()                 hot-swaping the shell
\end{lstlisting}

  \item Here an example, we print all clauses in the AAL program.
\begin{lstlisting}
shell> clauses()

/* Result :
kim_policy cloudX_policy 

*/
\end{lstlisting}

  \item self variable represent the co
\begin{lstlisting}
shell> self

/* Result :
<AALCompiler.AALCompilerListener object at 0x7f8b00ce8630>

*/

shell> man(self)

/* Result :
printing manual for <class 'AALCompiler.AALCompilerListener'>
Manual for aal compiler visitor
 - Attributes
   -  aalprog      Get the AAL program instance
   -  file         The AAL source file
   -  libs         Show the loaded libraries
   -  libsPath     Print the standard lib path
 - Methods
   - load_lib(lib_name)    Load an aal file
   - clause(clauseId)      Get a clause
   - show_clauses()        Print all clauses
   - get_clauses()         Get all clauses (array format)
   - get_macros()          Get all macros

*/

shell> man(aalprog)

/* Result :
printing manual for <class 'AALMetaModel.m_aalprog'>

    AAL program class.
    Note that clauses and macros extends a declarable type, but are not in the declarations dict

    Attributes
        - clauses: a list that contains all program clauses
        - declarations: a dictionary that contains lists of typed declarations
        - comments: a list that contains program's comment
        - macros: a list that contains program's macros declarations
        - macroCalls: a list that contains program's comment

*/
\end{lstlisting}

  \item hotswaping commands are used for debugging purpose only.
  r() command allows you to reload the shell after 

  hs(module) reloading other modules after 
  ! IMORTANT : to use hotswaping properly you must enable it explicitly in aalc arguments
  \texttt{-d / --hotswap},

\end{itemize}

}